\chapter{Transformations chimiques}
\section{\'Equation bilan}
Lorsque l'on écrit une équation bilan, il faut toujours veiller à respecter les lois de conservation de la matière :
\begin{itemize}
	\item Conservation de la masse et donc des atomes,
	\item Conservation de la charge électronique totale.
\end{itemize}

On peut écrire une équation bilan de plusieurs façons. La façon la plus classique est celle où les réactifs sont à gauche de l'équation et les produits à droite.
Une autre possibilité est de l'écrire comme cela :
$$\sum_i{\nu_i B_i} = 0$$
où $B_i$ est un composant et $\nu_i$ son coefficient \stoe algébrique, négatif si le composant est un réactif.

\notation{On notera les réactifs $\nu_i$ et $B_i$ et les produits $\nu_i'$ et $B_i'$}

En fait, on a :
$$\sum_i{\nu_iB_i} = 0 \Leftrightarrow \sum_i{\nu_iB_i} + \sum_i{\nu_i'B_i'} = 0 \Leftrightarrow \sum_i{-\nu_iB_i} = \sum_i{\nu_i'B_i'}$$
Par exemple :
\begin{center}
	$\ce{H2 + 1/2O2} = \ce{H2O} \Leftrightarrow \ce{H2O} - \ce{H2} - \ce{1/2O2} =0$
\end{center}
De même, pour les variations de grandeurs \thermo, on aura :
$$\Delta_\rsc G = \sum_i{\nu_i\Delta G_i}$$

\notation{On essaiera toujours d'avoir le plus petit des \coef s \stoe s égal à 1}
\attention{%
Dans une réaction chimique, il faut indiquer l'état physique des \glspl{reactant}, par exemple :
$$\ce{2H2_{(g)} + 1O2_{(g)} = 2H2O_{(l)}}$$%
}

\section{Proportions \stoes}

Soit la réaction générique et le tableau d'avancement suivants :
\begin{center}
	\begin{tabular}{cccccccc}
		&\reacStdTab{}\\[0.1cm]
		à $t=0$ & $n_1(0)$ && $n_2(0)$ && $n_1'(0)$ && $n_2'(0)$\\[0.1cm]
		à $t$ qcq & $n_1(t)$ && $n_2(t)$ && $n_1'(t)$ && $n_2'(t)$\\
	\end{tabular}
\end{center}
la réaction se produit dans des proportions \stoes si on a $\frac{n_1(0)}{|\nu_1|} = \frac{n_2(0)}{|\nu_2|}$ et $\frac{n_1'(0)}{|\nu_1'|} = \frac{n_2'(0)}{|\nu_2'|}$. Si ce n'est pas le cas, le \gls{reactant} pour lequel ce rapport est le plus petit est en défaut et celui pour lequel ce rapport est le plus grand est en excès.
\rem{Le \gls{reactant} en défaut est aussi appelé \gls{reactant} limitant}

\section{\'Evolution de la composition d'un système}
Dans le cas d'une \gls{reac_tot}, ou réaction quantitative, à l'équilibre le \gls{reactant} limitant n'existe plus qu'à l'état de traces.
\rem{On a $\Deltar G^0 = -RT\ln K_{eq}$ or%
$$K_{eq} = \prod_i{C_i^{\nui}}=\frac{\prod_i{C_i'^{\nui'}}}{\prod_i{C_i^{\nui}}}$$
donc, si $C_1=0$ alors $\Deltar G^0 = \infty$}

Dans le cas d'une \gls{reac_rev}, tous les \glspl{reactant} restent présent en quantités "visibles" à l'équilibre.

\subsection{\Gls{avancement} de la réaction $\xi$ ou $x$}

Soit la réaction 
Soit la réaction générique et le tableau d'avancement suivants :
\begin{center}
	\begin{tabular}{cccccccc}
		&\reacStdTab{}\\[0.1cm]
		à $t$ qcq & $n_1(t)$ && $n_2(t)$ && $n_1'(t)$ && $n_2'(t)$\\
	\end{tabular}
\end{center}
alors on a
$$\xi(t) = \frac{\Delta n_i}{\nui} = \frac{n_i(t) - n_i(0)}{\nui}$$
on en déduit $n_i(t) = n_i(0) + \nui \xi (t)$