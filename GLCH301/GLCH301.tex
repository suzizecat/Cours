\documentclass{book}
 
\usepackage[utf8]{inputenc}
\usepackage[T1]{fontenc}
\usepackage[french]{babel} 
\usepackage{lmodern} % Pour changer le pack de police
\usepackage{pageTitre}
\usepackage{glossaries}

\usepackage[version=3]{mhchem}%Pour les formules chimiques

%\usepackage{makeidx}
%\makeindex
\makenoidxglossaries
 
 \institut{\'Ecole Polytechnique Universitaire de Montpellier}
\departement{Département PEIP}
\titre{Cours de thermochimie}
\nom{Faucher}
\prenom{Julien}
\auteurParNom{}%
\sousTitre{HLCH301}
\addrDebut{UNIVERSITE MONTPELIER 2 SCIENCES ET TECHNIQUES DU LANGUEDOC}
\addrSuite{Place Eugène Bataillon -- 34095 MONTPELLIER -- CEDEX 5}
\tel{04 67 14 31 60}
\fax{04 67 14 45 14}
\mail{scola@polytech.univ-montp2.fr}

\newcommand{\ignore}[1]{}

\newcommand{\thermo}{thermodynamique}
\newcommand{\Thermo}{Thermodynamique}

\newcommand{\coef}{coefficient}
\newcommand{\stoe}{st\oe chiométrique}
\newcommand{\stoes}{st\oe chiométriques}

\newcommand{\nui}{\nu_i}
\newcommand{\Deltar}{\Delta_\textsc{r}}

\newcommand{\reacStdTab}{$\nu_1B_1$&+&$\nu2B_2$&$\rightleftharpoons$&$\nu_1'B_1'$&+&$\nu_2'B_2'$}

\newcommand{\notation}[1]{\paragraph{Notation :}#1\\}
\newcommand{\rem}[1]{\paragraph{Remarque :}#1\\}
\newcommand{\attention}[1]{\paragraph{Attention :}#1\\}

\newcommand{\rsc}{\textsc{r}}


\newglossaryentry{avancement}{name={avancement},description={\'Etat de la réaction à un instant donné}}

\newglossaryentry{coef_disso}{name={coefficient de dissociation},description={Taux de conversion pour un seul réactant}}

\newglossaryentry{reactant}{name={réactant},description={Composé entrant en jeu lors d'une réaction chimique}}
\newglossaryentry{reac_tot}{name={réaction totale},description={Réaction pour laquelle le réactant limitant n'existe plus qu'à l'état de traces}}
\newglossaryentry{reac_rev}{name={réaction réversible},description={Réaction pour laquelle tous les réactants sont présents à l'équilibre}}

\newglossaryentry{taux_conv}{name={taux de conversion},description={Proportion de réactant consommé à un temps $t$}}

\newglossaryentry{varDonder}{name={variable de \textsc{Donder}},description={Avancement élémentaire donné par $d\xi=\frac{dn_i}{\nu_i}$}}

\begin{document}
\titrePerso{}

\frontmatter 

\tableofcontents

\mainmatter
\chapter{Transformations chimiques}
\section{\'Equation bilan}
Lorsque l'on écrit une équation bilan, il faut toujours veiller à respecter les lois de conservation de la matière :
\begin{itemize}
	\item Conservation de la masse et donc des atomes,
	\item Conservation de la charge électronique totale.
\end{itemize}

On peut écrire une équation bilan de plusieurs façons. La façon la plus classique est celle où les réactifs sont à gauche de l'équation et les produits à droite.
Une autre possibilité est de l'écrire comme cela :
$$\sum_i{\nu_i B_i} = 0$$
où $B_i$ est un composant et $\nu_i$ son coefficient \stoe algébrique, négatif si le composant est un réactif.

\notation{On notera les réactifs $\nu_i$ et $B_i$ et les produits $\nu_i'$ et $B_i'$}

En fait, on a :
$$\sum_i{\nu_iB_i} = 0 \Leftrightarrow \sum_i{\nu_iB_i} + \sum_i{\nu_i'B_i'} = 0 \Leftrightarrow \sum_i{-\nu_iB_i} = \sum_i{\nu_i'B_i'}$$
Par exemple :
\begin{center}
	$\ce{H2 + 1/2O2} = \ce{H2O} \Leftrightarrow \ce{H2O} - \ce{H2} - \ce{1/2O2} =0$
\end{center}
De même, pour les variations de grandeurs \thermo, on aura :
$$\Delta_\rsc G = \sum_i{\nu_i\Delta G_i}$$

\notation{On essaiera toujours d'avoir le plus petit des \coef s \stoe s égal à 1}
\attention{%
Dans une réaction chimique, il faut indiquer l'état physique des \glspl{reactant}, par exemple :
$$\ce{2H2_{(g)} + 1O2_{(g)} = 2H2O_{(l)}}$$%
}

\section{Proportions \stoes}

Soit la réaction générique et le tableau d'avancement suivants :
\begin{center}
	\begin{tabular}{cccccccc}
		&\reacStdTab{}\\[0.1cm]
		à $t=0$ & $n_1(0)$ && $n_2(0)$ && $n_1'(0)$ && $n_2'(0)$\\[0.1cm]
		à $t$ qcq & $n_1(t)$ && $n_2(t)$ && $n_1'(t)$ && $n_2'(t)$\\
	\end{tabular}
\end{center}
la réaction se produit dans des proportions \stoes si on a $\frac{n_1(0)}{|\nu_1|} = \frac{n_2(0)}{|\nu_2|}$ et $\frac{n_1'(0)}{|\nu_1'|} = \frac{n_2'(0)}{|\nu_2'|}$. Si ce n'est pas le cas, le \gls{reactant} pour lequel ce rapport est le plus petit est en défaut et celui pour lequel ce rapport est le plus grand est en excès.
\rem{Le \gls{reactant} en défaut est aussi appelé \gls{reactant} limitant}

\section{\'Evolution de la composition d'un système}
Dans le cas d'une \gls{reac_tot}, ou réaction quantitative, à l'équilibre le \gls{reactant} limitant n'existe plus qu'à l'état de traces.
\rem{On a $\Deltar G^0 = -RT\ln K_{eq}$ or%
$$K_{eq} = \prod_i{C_i^{\nui}}=\frac{\prod_i{C_i'^{\nui'}}}{\prod_i{C_i^{\nui}}}$$
donc, si $C_1=0$ alors $\Deltar G^0 = \infty$}

Dans le cas d'une \gls{reac_rev}, tous les \glspl{reactant} restent présent en quantités "visibles" à l'équilibre.

\subsection{\Gls{avancement} de la réaction $\xi$ ou $x$}

Soit la réaction 
Soit la réaction générique et le tableau d'avancement suivants :
\begin{center}
	\begin{tabular}{cccccccc}
		&\reacStdTab{}\\[0.1cm]
		à $t$ qcq & $n_1(t)$ && $n_2(t)$ && $n_1'(t)$ && $n_2'(t)$\\
	\end{tabular}
\end{center}
alors on a
$$\xi(t) = \frac{\Delta n_i}{\nui} = \frac{n_i(t) - n_i(0)}{\nui}$$
on en déduit $n_i(t) = n_i(0) + \nui \xi (t)$

$\xi(t)$ est \emph{l'\gls{avancement} de la réaction} est est exprimé en \emph{mole d'avancement}\footnote{Différent de la mole de matière}. On définit également l'avancement élémentaire
$$d\xi=\frac{dn_i}{\nui}$$
aussi appelé \gls{varDonder}.

\subsection{Sens d'évolution}
On peut déterminer le sens d'évolution d'un système thermochimique via l'\gls{avancement}. En effet, 
\begin{itemize}
	\item Si $\xi(t) > 0$ ou $d\xi>0$ l'avancement est "positif", le système évolue donc dans le sens direct ou sens 1 ($\rightarrow$).
	\item Si $\xi(t) < 0$ ou $d\xi<0$ l'avancement est "négatif", le système évolue donc dans le sens indirect ou sens -1 ($\leftarrow$).
\end{itemize}

\subsection{Limites d'avancement}
On note $\avcinfty$ la valeur théorique de l'avancement lorsque tous les réactants de la partie gauche de l'équation de réaction sont consommés.
De même, on a $\avcinftym$ la valeur lorsque tous les réactants de la partie droite sont consommés.

Notons $t_f$ le temps tel que $t_f - t_0$ soit la durée de la réaction. On a :
\begin{itemize}
	\item Pour une réaction totale : $\xi(t_f) = \xi_{\pm\infty}$ (suivant le sens de la réaction)
	\item Pour un équilibre : $\xi(t_f) \in [\avcinftym ; \avcinfty]$
\end{itemize}

On peut facilement calculer la valeur de $\xi_{\pm\infty}$ en utilisant la formule de calcul de l'avancement sur le réactif limitant et avec $n_i(t_f)=0$.
$$\xi_{\pm\infty}=\frac{-n_i(0)}{|\nui|}$$

\subsection{Taux de conversion \& coefficient de dissociation}
Le \gls{taux_conv} $\tau$ représente la proportion de réactant consommé à un temps $t$ de la réaction. Il est défini par la formule :
$$\tau=\frac{n_i(0)-n_i(t)}{n_i(0)}$$
ou par
$$\tau=\frac{\xi(t)}{\xi_{\pm\infty}}$$

On doit toujours avoir $\tau\in[0;1]$. S'il n'y a qu'un seul réactif, le taux de conversion est appelé \emph{\gls{coef_disso}} et est noté $\alpha$

\backmatter
\printnoidxglossaries

\ignore{\appendix
  
Chapitres annexes
%\bibliographystyle{} % Le style est mis entre crochets.
%\bibliography{bibli} % Mon fichier de base de données s'appelle bibli.bib.

\backmatter

Epilogue


\listoffigures
\listoftables
\printindex
}
\end{document}