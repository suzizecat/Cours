\newcommand{\CC}{\mathbb{C}\xspace}
\newcommand{\IIma}{\mathbb{I}\xspace}
\newcommand{\KK}{\mathbb{K}\xspace}
\newcommand{\NN}{\mathbb{N}\xspace}
\newcommand{\QQ}{\mathbb{Q}\xspace}
\newcommand{\RR}{\mathbb{R}\xspace}



\newcommand{\II}{\text{I}\xspace}
\newcommand{\JJ}{\text{J}\xspace}
\newcommand{\VV}{\text{V}\xspace}

\newcommand{\CCr}{\mathscr{C}\xspace}

\renewcommand{\AA}{\mathcal{A}\xspace}
\newcommand{\BB}{\mathcal{B}\xspace}
\newcommand{\oo}{o}


\newcommand{\der}{\,\mathrm{d}}

\renewcommand{\epsilon}{\varepsilon}
\renewcommand{\phi}{\varphi}

\newcommand{\tq}{\text{~tq~}}
\newcommand{\BW}{Bolzano-Weierstrass}

\newcommand{\pinfty}{+\infty}
\newcommand{\minfty}{-\infty}
\newcommand{\pminfty}{\pm\infty}

\theoremstyle{plain}
\newtheorem{thm}{Théorème}[chapter]
\newtheorem{axio}[thm]{Axiome}
\newtheorem{propo}[thm]{Proposition}
\newtheorem{lem}[thm]{Lemme}
\newtheorem{coro}[thm]{Corollaire}

\theoremstyle{definition}
\newtheorem{defi}[thm]{Definition}
\newtheorem{prop}[thm]{Propriété}

\theoremstyle{remark}
\newtheorem*{expl}{Exemple}
\newtheorem*{idee}{Idée}
\newtheorem*{inter}{Interêt}
\newtheorem*{nota}{Notation}
\newtheorem*{note}{Note}
\newtheorem*{rem}{Remarque}
\newtheorem*{appl}{Application}
