\chapter{Suites}
Dans ce chapitre, on pose $\KK = \RR$ ou $\CC$ ou $\QQ$

\section{Propriété fondamentale de $\RR$}
Soit $\AA \subset \RR$.

\begin{defi}
On dit que $M \in \RR$ est un majorant / minorant de $\AA$ si $\forall x \in \AA$, $x \leq M$ / $x \geq M$
\end{defi}
\begin{defi}
On dit que $M$ est une borne supérieure/inférieure de $\AA$ si :
\begin{itemize}
	\item $M$ est majorant/minorant de $\AA$
	\item Si $M'$ est majorant/minorant de $\AA$, on doit avoir $M'>M$ / $M'<M$. C'est à dire que $M$ doit être le plus petit/grand majorant/minorant de $\AA$
\end{itemize}
\end{defi}
\begin{prop}
Si $\AA$ admet une borne supérieure ou inférieure, cette borne est unique.
\end{prop}

\begin{proof}

Soient $M_1$ et $M_2$ deux bornes supérieures de $\AA$.\\
\begin{tabular}{r!{$\Rightarrow$}l}
	Alors $M_1$ majore $\AA$ & $M_1 \geq M_2$\\
	$M_2$ majore $\AA$ & $M_2 \geq M_1$
\end{tabular}\\
Donc $M_1=M_2$.
\end{proof}

\begin{nota}
On note une borne supérieure $\sup \AA$ et une borne inférieure $\inf \AA$.
\end{nota}

\begin{rem}
On se place dans $\KK = \QQ$ et $\AA = \{x\in \QQ, x^2\leq 2\}$\\
Alors on a $A =[-\sqrt{2};\sqrt{2}] \cap \QQ$. Dans ce cas, $\AA$ n'a pas de bornes dans $\QQ$. En effet, si ces bornes existent, elles valent $\pm \sqrt{2} \notin \QQ$. C'est une des raisons de la création de l'ensemble des réels.
\end{rem}

\begin{axio}
Soit $\AA \subset \RR$ tel que $\AA \neq \emptyset$ et $\AA$ majorée. Alors $\sup \AA \in \RR$ existe.
\end{axio}

\section{Suites}
\begin{defi}
Une suite de $\KK$ est une application de $\NN$ dans $\KK$\\
\begin{tabular}{rcl}
$u:\NN$ &$\longmapsto$&$\KK$\\
$n$&$\rightarrow$&$u(n)$
\end{tabular}
\end{defi}
\begin{nota}
$u(n)$ est noté $u_n$ et la suite $u$ est notée $(u_n)_{n\in \NN}$ ou simplement $u_n$.
\end{nota}
$u_n = \ln\left(1+\frac{1}{n}\right)$

\begin{rem}
On peut dire qu'une suite est une restriction à $\NN$ d'une fonction $f:[0~;+\infty[\longrightarrow\KK$
\end{rem}

\begin{axio}[Récurrence]
Soit $P(n)$ une propriété dépendant de $n\in \NN$. Si
\begin{itemize}
	\item il existe $x$ tel que $P(x)$ est vérifiée,
	\item pour tout $n>x$, $P(n)$ nous permet de déduire $P(n+1)$
\end{itemize}
alors, $P(n)$ est vérifiée pour tout $n>x$.
\end{axio}

\begin{prop}\end{prop}

\begin{proof}\end{proof}

\begin{defi}[Monotonie]
Pour $\KK \neq\CC$, soit $(u_n)$ une suite de $\KK$, alors $(u_n)$ est croissante si 
$$\forall p,q \in \NN, p \leq q \Rightarrow u_p \leq u_q$$
\end{defi}
\section{Convergence}
\begin{defi}
Soit $(u_n)$ une suite de $\KK$ et $l\in\KK$. On dit que $(u_n)$ tend vers $l$ si :
$$\forall\epsilon>0,\exists N\in\NN\text{~tq~}\forall n\geq N, |u_n-l| < \epsilon$$
\end{defi}
\begin{defi}
Pour $\KK=\RR$, on dit que ($u_n$) tend vers l'infini si et seulement si :
$$\forall A>0,\exists N\in\NN\tq\forall n,n>N, u_n>A$$
\end{defi}
\begin{defi}
On dit que $(u_n)$ converge si $(u_n)$ admet une limite dans $\KK$. En particulier, une suite tendant vers l'infini diverge\footnote{Ne converge pas}.
\end{defi}

\begin{nota}
\begin{align*}
  &\lim u_n = l\\
\Leftrightarrow& \lim\limits_{\pinfty}u_n=l\\
\Leftrightarrow& \lim\limits_{n\to\pinfty}u_n = l\\
\Leftrightarrow& u_n \rightarrow l\\
\Leftrightarrow& u_n \underset{n\to\pinfty}{\longrightarrow}l
\end{align*}
\end{nota}

\begin{thm}
Soit $\KK=\RR$. Soit $(u_n)$ une suite de $\KK$ telle que $(u_n)$ est majorée et croissante. Alors $(u_n)$ converge et $\lim u_n = \sup u_n$ ($n\in\NN$)
\end{thm}

\begin{appl}[Suites Adjacentes]
Soit $(u_n)$ et $(v_n)$ deux suites dans $\RR$ telles que :
\begin{enumerate}
	\item $(u_n)$ est croissante et $(v_n)$ est décroissante.
	\item $\lim v_n-u_n = 0$
\end{enumerate}
Alors $\forall n, u_n >v_n$ et $(v_n)$ et $(u_n)$ convergent vers la même limite.
\end{appl}
\section{Suites extraites, \BW}
\begin{defi}
Soient $(u_n)$ et $(v_n)$ deux suites de $\KK$. On dit que $(v_n)$ est extraite de $(u_n)$ s'il existe $\phi : \NN\to\NN$ strictement croissante telle que pour tout $n$, $v_n = u_{\phi(n)}$
\end{defi}
\begin{nota}
$v_k =u_{n_k}$. On dit aussi que $(v_n)$ est une sous-suite de $(u_n)$.
\end{nota}
\begin{rem}
~
\begin{enumerate}
	\item Toute suite $(u_n)$ est extraite d'elle même. Il suffit de prendre $\phi(n)=n$.
	\item Soit $(u_n) = (-1)^n$ et $\phi(n)=2n$. On a alors $v_n=u_{\phi(n)} = u_{2n} = 1$.\\Ou, si $\phi(n)=2n+1$, $v_n=u_{\phi(n)} = u_{2n+1} = -1$. 
\end{enumerate}
\end{rem}

\begin{lem} Soit $(u_n)$ une suite de $\KK$. Alors : $u_n\to l$ $\Leftrightarrow$ pour toute suite $(v_n)$ extraite de $(u_n)$, $v_n\to l$
\end{lem}
\begin{proof}\end{proof} .

\begin{thm}[Théorème de \BW]
Soit $(u_n)$ une suite bornée (minorée et majorée). Il existe au moins une suite $(v_n)$ extraite de $(u_n)$ convergente.
\end{thm}
\section{Le critère de Cauchy}
\begin{inter}
Le critère de Cauchy permet de montrer qu'une suite converge sans connaître sa limite et même sans savoir à priori s'il y en a une.
\end{inter}
\begin{defi}
Soit $(u_n)$ une suite de $\KK$. On dit que $(u_n)$ est une suite de Cauchy si :
$$\forall \epsilon>0\,,\exists N\tq\forall\, p,\,q \text{~si~} p,\,q>N,\,|u_p-i_q|<\epsilon$$
\end{defi}
\begin{prop}
Si $(u_n)$ est convergente, $(u_n)$ est de Cauchy
\end{prop}

\begin{rem}
Pour $\KK=\QQ$, prenons $\alpha_n\in\QQ\tq\alpha_n\to\sqrt{2}$. Alors, $(\alpha_n)$ est de Cauchy mais $\sqrt{2}\notin\QQ$ donc $(\alpha_n)$ ne converge pas dans $\QQ$ et la réciproque de Cauchy est fausse dans $\QQ$. C'est d'ailleurs la raison historique de la création de $\RR$
\end{rem}

\begin{lem}
Toute suite de Cauchy est bornée
\end{lem}
\begin{proof}
Soit $(u_n)$ une suite de Cauchy. On prend $\epsilon=1$. Il existe N tel que pout $p,\,q\geq N,\,|u_p-u_q|<1$ et notamment si $q=N$.

\begin{tabular}{rcl}
Soit $M$&=&$\max(u_0,u_1,\ldots,u_{N-1},u_N+1)$\\
	$m$&=&$\min(u_0,u_1,\ldots,u_{N-1},u_N-1)$\\
\end{tabular}\\
Alors, $M$ majore $(u_n)$ et $m$ minore $(u_n)$
\end{proof}

\begin{lem}
Soit $(u_n)$ une suite de Cauchy telle qu'il existe une sous suite convergente $(u_{\phi(n)})$ ou $(v_n)$, alors $(u_n)$ converge.
\end{lem}
\begin{proof}

\end{proof}

\begin{thm}[Théorème de Cauchy]
Pour $\KK = \CC\text{~ou~}\RR$. Soit $(u_n)$ une suite de $\KK$.\\
\begin{center}
	$(u_n)$ converge $\Leftrightarrow$ $(u_n)$ est de Cauchy
\end{center}
\end{thm}
