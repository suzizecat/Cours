\chapter{Suites}
Dans ce chapitre, on pose $\KK = \RR$ ou $\CC$ ou $\QQ$

\section{Propriété fondamentale de $\RR$}
Soit $\AA \subset \RR$.
\definition{}
On dit que $M \in \RR$ est un majorant / minorant de $\AA$ si $\forall x \in \AA$, $x \leq M$ / $x \geq M$

\definition{}
On dit que $M$ est une borne supérieure/inférieure de $\AA$ si :
\begin{itemize}
	\item $M$ est majorant/minorant de $\AA$
	\item Si $M'$ est majorant/minorant de $\AA$, on doit avoir $M'>M$ / $M'<M$. C'est à dire que $M$ doit être le plus petit/grand majorant/minorant de $\AA$
\end{itemize}

\prop{}
Si $\AA$ admet une borne supérieure ou inférieure, cette borne est unique.

\demonstration{}
Soient $M_1$ et $M_2$ deux bornes supérieures de $\AA$.\\
\begin{tabular}{r!{$\Rightarrow$}l}
	Alors $M_1$ majore $\AA$ & $M_1 \geq M_2$\\
	$M_2$ majore $\AA$ & $M_2 \geq M_1$
\end{tabular}\\
Donc $M_1=M_2$.

\notation{}
On note une borne supérieure $\sup \AA$ et une borne inférieure $\inf \AA$.

\remarque{}
On se place dans $\KK = \QQ$ et $\AA = \{x\in \QQ, x^2\leq 2\}$\\
Alors on a $A =[-\sqrt{2};\sqrt{2}] \cap \QQ$. Dans ce cas, $\AA$ n'a pas de bornes dans $\QQ$. En effet, si ces bornes existent, elles valent $\pm \sqrt{2} \notin \QQ$. C'est une des raisons de la création de l'ensemble des réels.

\axiome{}
Soit $\AA \subset \RR$ tel que $\AA \neq \emptyset$ et $\AA$ majorée. Alors $\sup \AA \in \RR$ existe.

\section{Suites}
\definition{}
Une suite de $\KK$ est une application de $\NN$ dans $\KK$\\
\begin{tabular}{rcl}
$u:\NN$ &$\longmapsto$&$\KK$\\
$n$&$\rightarrow$&$u(n)$
\end{tabular}

\notation{}
$u(n)$ est noté $u_n$ et la suite $u$ est notée $(u_n)_{n\in \NN}$ ou simplement $u_n$.

\exemple{}
$u_n = \ln\left(1+\frac{1}{n}\right)$

\remarque{}
On peut dire qu'une suite est une restriction à $\NN$ d'une fonction $f:[0~;+\infty[\longrightarrow\KK$

\axiome{Récurrence}
Soit $P(n)$ une propriété dépendant de $n\in \NN$. Si
\begin{itemize}
	\item il existe $x$ tel que $P(x)$ est vérifiée,
	\item pour tout $n>x$, $P(n)$ nous permet de déduire $P(n+1)$
\end{itemize}
alors, $P(n)$ est vérifiée pour tout $n>x$.

\prop{}

\demonstration{}

\definition{Monotonie}
Pour $\KK \neq\CC$, soit $(u_n)$ une suite de $\KK$, alors $(u_n)$ est croissante si 
$$\forall p,q \in \NN, p \leq q \Rightarrow u_p \leq u_q$$