\chapter{Comparaison de fonctions et développements limités}
\section{Négligeabilité  et équivalence}

\paragraph{Définition (Abus) :}
Soit $a\in\RR$. Un voisinage de $a$ est un intervalle de la forme $]a-\delta;a+\delta[$ ou $]a-\delta;a+\delta[~/\{0\}$ avec $\delta>0$

\definition{Négligeabilité}
Soit $f,g:\II\subset\RR \to \RR$ deux fonctions.\\
On dit que $f$ est négligeable devant $g$ en $a\in\RR$ s'il existe un voisinage $\VV$ de $a$ et une fonction $\epsilon:\VV\to\RR$ telle que :
\begin{itemize}
	\item $f(x) = g(x)\epsilon (x)$ pour $x\in \VV$
	\item $\lim\limits_{a}\epsilon(x)=0$
\end{itemize}
\notation{}
Si $f$ est négligeable devant $g$, on note $f \ll_{a}g$ (Physique) ou $f=o_a(g)$ (Maths)

\remarque{}
Si $g$ ne s'annule pas sur $\VV$, $f \ll_a g \Leftrightarrow \lim\limits_{x\to a}\frac{f(x)}{g(x)}=0$

\exemple{}

\definition{Équivalence}
Soit $f,g:\II\subset\RR \to \RR$ deux fonctions. Soit $a\in\RR$.\\
On dit que $f$ est équivalente à $g$ en $a$ si on a :
$$f(x) = g(x) + o_a(g(x))$$
C'est à dire si $f=g+$ quelque chose de négligeable devant $g$.

\remarque{}
\begin{eqnarray*}
		f(x) &=& g(x) + o_a(g(x))\\
		\Leftrightarrow f(x) & = & g(x) + g(x)\epsilon(x)\\
		\Leftrightarrow f(x) & = & g(x)(1+\epsilon(x))\\
		\Leftrightarrow \frac{f(x)}{g(x)} &=& 1 + \epsilon(x)\\
		\Leftrightarrow \lim\limits_a\frac{f(x)}{g(x)} &=& 1\\
\end{eqnarray*}
Les deux dernières notations ne sont valides que si $g\neq 0$ au voisinage de $a$. $\epsilon$ est une fonction telle que $\lim\limits_a\epsilon=0$ 

\notation{}
$f$ est équivalente à $g$ en $a$ s'écrit $f\sim_{a}g$

\remarque{}
Si $f\sim_{a}g$ et $\lim\limits_ag=l$ alors,
\begin{eqnarray*}
		\lim\limits_a f(x) &=& \lim\limits_a g(x) *\lim\limits_a(1+\epsilon(x))\\
		& = & \lim\limits_a g(x)\\
		&=&l\\
\end{eqnarray*}

\proposition{}
Si $f\sim_ag$ et si $\lim\limits_ag$ existe, alors $\lim\limits_af=\lim\limits_ag$. \emph{Attention : la réciproque est fausse !}

\demonstration{}
\exemple{}

\section{Développements limités}
\idee{}
On va faire l'approximation de fonctions par des polynômes.

\definition{}
Soit $f:\II \subset \RR \to \RR$ et $a\in \II$.\\
On dira que $f$ admet un développement limité d'ordre $n$ en $a$ (noté $DL_n(a)$) s'il existe un polynôme $P$ de degré $n$ tel qu'au voisinage de a,
$$f(x) = P(x-a) + o_a((x-a)^n)$$

\remarque{}
On a $f(x+a) = P(x) + o_0(x^n)$ donc on fera les développements limités en 0

\prop{}
Si $f$ admet un $DL_n(0)$ alors, de développement limité est unique.

\demonstration{}
\exemple{}

\thm{Formule de Taylor}
Soit $f$ définie au voisinage de 0 et de classe $\CCr^n$