\chapter{Comparaison de fonctions et développements limités}
\section{Négligeabilité  et équivalence}

\defi
Soit $a\in\RR$. Un voisinage de $a$ est un intervalle de la forme $]a-\delta;a+\delta[$ ou $]a-\delta;a+\delta[~/\{0\}$ avec $\delta>0$

\defi[Négligeabilité]
Soit $f,g:\II\subset\RR \to \RR$ deux fonctions.\\
On dit que $f$ est négligeable devant $g$ en $a\in\RR$ s'il existe un voisinage $\VV$ de $a$ et une fonction $\epsilon:\VV\to\RR$ telle que :
\begin{itemize}
	\item $f(x) = g(x)\epsilon (x)$ pour $x\in \VV$
	\item $\lim\limits_{a}\epsilon(x)=0$
\end{itemize}

\nota
Si $f$ est négligeable devant $g$, on note $f \ll_{a}g$ (Physique) ou $f=o_a(g)$ (Maths)

\rem
Si $g$ ne s'annule pas sur $\VV$, $f \ll_a g \Leftrightarrow \lim\limits_{x\to a}\frac{f(x)}{g(x)}=0$

\expl


\defi[Équivalence]
Soit $f,g:\II\subset\RR \to \RR$ deux fonctions. Soit $a\in\RR$.\\
On dit que $f$ est équivalente à $g$ en $a$ si on a :
$$f(x) = g(x) + o_a(g(x))$$
C'est à dire si $f=g+$ quelque chose de négligeable devant $g$.

\rem
\begin{eqnarray*}
		f(x) &=& g(x) + o_a(g(x))\\
		\Leftrightarrow f(x) & = & g(x) + g(x)\epsilon(x)\\
		\Leftrightarrow f(x) & = & g(x)(1+\epsilon(x))\\
		\Leftrightarrow \frac{f(x)}{g(x)} &=& 1 + \epsilon(x)\\
		\Leftrightarrow \lim\limits_a\frac{f(x)}{g(x)} &=& 1\\
\end{eqnarray*}itemize
Les deux dernières notations ne sont valides que si $g\neq 0$ au voisinage de $a$. $\epsilon$ est une fonction telle que $\lim\limits_a\epsilon=0$ 

\nota
$f$ est équivalente à $g$ en $a$ s'écrit $f\sim_{a}g$

\rem
Si $f\sim_{a}g$ et $\lim\limits_ag=l$ alors,
\begin{eqnarray*}
		\lim\limits_a f(x) &=& \lim\limits_a g(x) *\lim\limits_a(1+\epsilon(x))\\
		& = & \lim\limits_a g(x)\\
		&=&l\\
\end{eqnarray*}

\propo{}
Si $f\sim_ag$ et si $\lim\limits_ag$ existe, alors $\lim\limits_af=\lim\limits_ag$. \emph{Attention : la réciproque est fausse~!}

\begin{proof}\end{proof}{}
\expl.

\section{Développements limités}
\idee
On va faire l'approximation de fonctions par des polynômes.

\defi
Soit $f:\II \subset \RR \to \RR$ et $a\in \II$.\\
On dira que $f$ admet un développement limité d'ordre $n$ en $a$ (noté $DL_n(a)$) s'il existe un polynôme $P$ de degré $n$ tel qu'au voisinage de a,
$$f(x) = P(x-a) + o_a((x-a)^n)$$

\rem
On a $f(x+a) = P(x) + o_0(x^n)$ donc on fera les développements limités en 0

\prop{}
Si $f$ admet un $DL_n(0)$ alors, de développement limité est unique.

\begin{proof}\end{proof}
\expl

\thm{Formule de Taylor}
Soit $f$ définie au voisinage de 0 et de classe $\CCr^n$ ($n$ fois dérivable avec $f^{(n)}=\frac{d^nf(0)}{dx^n}$ continue). On a :
$$f(x)=\sum_{k=0}^n{\left(\frac{f^{(n)}(0)}{k!}x^k\right)}+o(x^n)$$

\begin{proof}\end{proof}
Voire Wikipédia.

\coro
Si $f$ est $\CCr^n$ alors elle admet un $DL_n(0)$

\expl
Posons $f(x)=e^x$. $\forall n,f(x)^{(n)} = e^x$ et $f(0)^{(n)} = 1$. On a donc
$$e^x = \sum_{k=0}^n{\frac{1}{k!}x^n}+o(x^n)$$
Soit
\[e^x=1+x+\frac{x^2}{2!}+\frac{x^3}{3!}+\cdots+\frac{x^n}{n!}+o(x^n)\]

\rem
C'est \emph{LA} bonne définition de $e^n$.

\expl

\rem
Taylor c'est bien, mais parfois très complexe (par exemple le $DL_{10}(0)$ de $\frac{1+x}{1-x^2}$

\rem
Un développement limité est une \emph{égalité} et non une approximation.

\prop[Opérations sur les $DL$ :]
Soit $f$ et $g$ deux fonctions. Admettons un $DL_n(0)$ tel que :
\begin{itemize}
	\item $f(x)=P(x)+o(x^n)$
	\item $g(x)=Q(x)+o(x^n)$
\end{itemize}
Avec $\deg P =\deg Q=n$. On a alors :
\begin{description}
	\item[Addition :]$f(x) + g(x) = P(x) + Q(x) + o(x^n)$
	\item[Produit :]$f(x) * g(x) = R(x) + o(x^n)$ où $R(x)$ est le polynôme $P(x)Q(x)$ tronqué à l'ordre $n$
	\item[Composition :]$f\circ g(x) = T(x) + o(n)$ où $T(x)$ est le polynôme $P(x)\circ Q(x)$ ($\deg R(x) = n^2$) tronqué au rang $n$.
	\item[Dérivation :] Si $f$ est dérivable, $f'(x) = P'(x) + o(x^{n-1})$
	\item[Intégration :]Si $f$ est continue, $F(x) = \int_0^x{f(t) dt} =\int_0^x{P(t)dt}+o(x^{n+1})$
\end{description}

\prop{}
Si $f(x) = P(x)+o(x^n)$ est un $DL_n(0)$ de $f$ avec $\deg P=n$, alors $P \sim_0 f$
