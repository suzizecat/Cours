\chapter{Suites et séries de fonctions}
\section{Norme infinie}
Soit $\II$ un intervalle de $\KK$ et $\BB(\II) = \{f:\II\to\KK,f\text{ bornée}\}$ l'ensemble des fonctions de $\II$ dans $\KK$ bornées. On remarque que $\BB(\II)$ est un espace vectoriel. On rappelle que "\textit{$f$ est bornée}" signifie qu'il existe une valeur $M$ telle que $\forall x\in\II, |f(x)| < M$

\begin{defi}[Norme infinie]
	La norme infinie de $f\in\BB(\II)$ est la valeur maximale que prend la valeur absolue de $f(x)$ pour $x\in\II$. Elle est notée
	$$||f||_\infty=\sup\limits_{x\in\II}|f(x)|$$
\end{defi}

\begin{prop}~
\begin{enumerate}
	\item $||f(x)||_\infty = 0$ signifie que, pour tout $x$ appartenant à $\II$, $f(x)=0$
    \item Pour $\lambda\in\KK$, $||\lambda f(x)||_\infty = |\lambda|*||f(x)||_\infty$.
    \item $||f\circ g||_\infty \leq ||f||_\infty + ||g||_\infty$
\end{enumerate}
\end{prop}

\begin{rem}
La norme infinie dépend de l'ensemble $\II$ su lequel on travaille. S'il y a une ambiguïté, on écrira $||f(x)||_{\infty ,\II}$.
\end{rem}

\begin{expl}
Si $\JJ$ appartient à $\II$, alors $||f(x)||_{\infty ,\JJ} \leq ||f(x)||_{\infty , \II}$
\end{expl}

\section{Convergence simple et convergence uniforme}
Pour un intervalle $\II$ quelconque appartenant à $\KK$, on se donne une fonction $f$ telle que :
\begin{center}
\begin{tabular}{rr!{$\to$}l}
f:&$\II$&$\KK$\\
&$x$&$f_n(x)$
\end{tabular}
\end{center}

\begin{defi}[Convergence simple]
On dit que la suite $(f_n)$ converge \emph{simplement} vers $f:\II\to\KK$ si, pour tout $x$, on a :
$$\lim\limits_{n\to\pinfty}f_n(x)=f(x)$$
\end{defi}

\begin{defi}[Convergence uniforme]\label{def_cvu}
On dit que $(f_n)$ converge \emph{uniformément} si $f_n-f$ est bornée et si :
$$\lim\limits_{n\to\pinfty}||f_n-f||_\infty = 0$$
\end{defi}

\begin{rem}
Pour tout $x$ appartenant à $\II$ :
$$0\leq |f_n(x)-f(x)|\leq \|f_n-f\|_\infty$$
Donc, si on a convergence uniforme, alors on a convergence simple.
\end{rem}

\begin{note}[Explications sur la convergence uniforme]
L'expression donnée ci-dessous n'est pas toujours très claire. Pour détailler, on peut dire que :
\begin{itemize}
	\item Nous sommes en présence de deux variables : $n$ et $x$.
    \item Lorsque l'on travaille sur \emph{la fonction $f(x)$}, la variable est $x$. Lorsque l'on travaille sur la suite $(f_n)$, la variable est $n$.
    \item La notation $f(x)$ désigne $f_n(x)$ lorsque $n$ tend vers l'infini.
\end{itemize}
Si l'on explicite au maximum la définition \ref{def_cvu} de la convergence uniforme, on obtient ceci :

On dit que la suite $(f_n(x))$ converge uniformément si on a :
$$\lim\limits_{n\to\infty}\left[\left\|f_n(x) - \lim\limits_{n\to\infty}\left[f_n(x)\right] \right\|_{\infty,\II} \right]=0$$
Dans cette expression, la valeur de $x$ est en fait fixée. En effet, l'utilisation de la norme infinie force une (ou des) valeurs de $x$ pour lesquelles l'expression $f_n-f$ est maximale. 

La convergence uniforme montre que, pour une suite de fonction donnée et pour une valeur de $x$ telle que $f_n(x) - f_\infty(x)$ est maximal, lorsque $n$ tend vers l'infini, $f_n$ tend vers $f_\infty$, indépendamment de $x$.

On peut donc dire que la convergence uniforme est la généralisation de la convergence simple à un intervalle $\II$ donné.
\end{note}

\begin{note}[Détermination d'une convergence uniforme]
Pour déterminer si une suite de fonction converge uniformément, il faut :
\begin{enumerate}
	\item Trouver une borne maximale, notée $\epsilon(n)$ à l'expression $|f_n(x)-f(x)|$. Cette borne doit dépendre de $n$
    \item Montrer que $\lim\limits_{n\to\infty}\epsilon(n) = 0$ indépendamment des valeurs de $x$
\end{enumerate}
\end{note}

\begin{prop}[Critère de Cauchy uniforme]
$(f_n)$ converge uniformément si elle est uniformément de Cauchy
$$\forall\epsilon>0,\exists N \tq \forall p,\,q\,,\,p,q\geq N \Rightarrow \|f_p-f_q\|_\infty < \epsilon$$
\end{prop}

\section{Inversion des bornes d'integrales}
\section{Convergence uniforme sur tout compact}
\begin{defi}[Approximation]
Un compact peut être considéré comme un intervalle borné $[a;b]$.
\end{defi}
\begin{nota}
On note la Convergence Uniforme sur tout Compact CVUK.
\end{nota}

\begin{defi}[Convergence uniforme sur tout compact]
Soit $f_n:\II\to\KK$ une suite de fonctions. On dit que $f_n$ converge uniforméménet sur tout compact vers $f:\II\to\KK$ si pour tout intervalle $[a;b]\subset\II$, la suite $f_n$ converge uniformément vers $f$ en $[a;b]$
\end{defi}
\begin{prop}
	\begin{enumerate}
		\item Si $f_n\toCVUK\KK$
		\item Si $f_n$ est $\CCr^1$, $f_n(x_0)\to\alpha$ et si $f'_n\toCVUK g$ alors il existe $f$, $f_n\toCVUK f$ et $f'=g$
	\end{enumerate}
\end{prop}

\begin{rem}
	CVU $\underset{\nLeftarrow}{\Rightarrow}$ CVUK $\underset{\nLeftarrow}{\Rightarrow}$ continuité
\end{rem}

\section{Séries de fonctions}
\begin{defi}
Une série de fonction est une suite de fonctions $S_n:\II\to\KK$ telle qu'il existe une suite de fonctions $(u_n):\II\to\KK$ avec $S_n(x)=\sum\limits_{k=0}^{n}u_k(x)$
\end{defi}
\begin{rem}[Conséquence immédiate]
	Tout le début du chapitre s'applique aux séries de fonctions. Cependant, quels sont les critères permetant de déterminer la convergence de $S_n$ ?
\end{rem}
\begin{propo}
	Si $S_n$ converge uniformément, alors $u_k$ converge uniformément vers 0.
\end{propo}
\begin{proof}
\end{proof}
Il y a deux critères impliquant la convergence uniforme :
\subsection{Convergence normale}
\begin{defi}
On dit que $S_n$ converge normalement\footnote{Convergence normale : CVN} si la série $\sum\|u_k\|_\infty$ converge.
\end{defi}

\begin{inter}
Les séries de type $\sum\|u_n\|_\infty$ sont des séries à termes positifs. Tous les critères vus en partie \ref{serie_th_pos} sont donc valides.
\end{inter}
\begin{thm}
La convergence normale d'une série entraine sa convergence uniforme. C'est à dire :
 si $\sum\|u_k\|_\infty$, alors $\sum u_k(x)$ converge uniformément.
\end{thm}

\begin{proof}\end{proof}

\subsection{Critère d'Abel uniforme}
\begin{thm}[d'Abel uniforme]
Soient $u_n(x)$ et $v_n(x)$ deux suites de fonctions de $\II$ dans $\RR$ telles que :
\begin{enumerate}
	\item $u_n(x)$ converge uniformément vers 0.
	\item $\sum\|u_{n+1}-u_n\|_\infty$ converge.
	\item $B_n(x)=\sum\limits_{p=0}^nv_p(x)$ est uniformément bornée\footnote{C'est à dire qu'il existe un réel $M$ tel que pour tout $x$ et tout $n$, $|B_n(x)|>M$}.
\end{enumerate}
Alors $\sum\left[u_n(x)*v_n(x)\right]$ converge uniformément.
\end{thm}

\section{Fonction $\zeta$}
\begin{defi}
Soit $z\in\CC$. La fonction $\zeta(z)$ est définie par :
$$\zeta(z)=\sum\limits_{n=1}^\infty \frac{1}{n^z}$$
\end{defi}
\begin{note}
Ici, $z$ est un nombre fixé. La fonction $\zeta$ est donc bien une fonction au sens courant du terme et sa valeur ne dépend que de $z$.
\end{note}
\begin{propo}[Convergence de $\zeta$]
 Si $\Re(z) > 1$, alors $\zeta(z)$ converge absolument. 
\end{propo}
\begin{proof}
Notons que $z$ est de la forme $x+iy$. Si on travaille sur la valeur absolue, ou le module dans l'ensemble des complexes, on obtient :
\begin{eqnarray}
	\left|\frac{1}{n^z}\right| &=&\left|\frac{1}{e^{z\ln(n)}}\right|\nonumber\\
	&=&\left|\frac{1}{e^{x\ln(n)}e^{iy\ln(n)}}\right|\label{complx_complet}\\
	&=&\frac{1}{e^{x\ln(n)}}\label{complx_simple}\\
	&=&\frac{1}{n^x}\nonumber\\
	&=&\frac{1}{n^{\Re(z)}}\label{eq_fin}
\end{eqnarray}
Notons que, pour le passage de l'équation \eqref{complx_complet} à l'équation \eqref{complx_simple}, on utilise le fait que le module d'un complexe de la forme $ae^{i\theta}$ vaut $a$ avec ici $a=e^{x\ln(n)}$.

On déduit de l'équation \eqref{eq_fin} (série de Riemann) que $|\zeta|$ converge, donc $\zeta$ converge absolument.
\end{proof}
