\chapter{Suites et séries de fonctions}
\section{Norme infinie}
Soit $\II$ un intervalle de $\KK$ et $\BB(\II) = \{f:\II\to\KK,f\text{ bornée}\}$ l'ensemble des fonctions de $\II$ dans $\KK$ bornées. On remarque que $\BB(\II)$ est un espace vectoriel. On rappelle que "\textit{$f$ est bornée}" signifie qu'il existe une valeur $M$ telle que $\forall x\in\II, |f(x)| < M$

\begin{defi}[Norme infinie]
	La norme infinie de $f\in\BB(\II)$ est la valeur maximale que prend la valeur absolue de $f(x)$ pour $x\in\II$. Elle est notée
	$$||f||_\infty=\sup\limits_{x\in\II}|f(x)|$$
\end{defi}
