\documentclass[a4paper]{book}

\usepackage[latin1]{inputenc}
\usepackage[T1]{fontenc}
\usepackage[french]{babel}
\usepackage[top=2cm, bottom=2cm, left=3.5cm, right=3.5cm]{geometry}
%\usepackage{layout}
%\usepackage{geometry}
%\usepackage{setspace}
\usepackage{soul}
\usepackage{textcomp}
\usepackage{ulem}
%\usepackage{eurosym}
%\usepackage{bookman}
%\usepackage{charter}
%\usepackage{newcent}
\usepackage{lmodern}
%\usepackage{mathpazo}
%\usepackage{mathptmx}
%\usepackage{url}
%\usepackage{verbatim}
%\usepackage{moreverb}
%\usepackage{listings}
%\usepackage{fancyhdr}
%\usepackage{wrapfig}
\usepackage{color}
%\usepackage{colortbl}
\usepackage{amsmath}
\usepackage{amssymb}
\usepackage{mathrsfs}
\usepackage{mathenv}
%\usepackage{asmthm}
\usepackage{makeidx}
\usepackage{graphicx}

%\graphicspath{{img/}}

\makeindex
\title{Cours de thermodynamique}
\author{Julien \textsc{Faucher}}
\date{\today}  
 
\begin{document}
 
\maketitle % Page de garde

\renewcommand{\contentsname}{Sommaire}
\setcounter{tocdepth}{2}
\tableofcontents
%\frontmatter 
%Pages introductives

\mainmatter

\part{GLPH101}
\chapter{Notions fondamentales et �nergie thermique.}
\section{Syst�me thermodynamique}
Un syst�me thermodynamique est compos� de mati�re et d'�nergie r�partis dans un volume de l'espace d�limit� par une surface ($\Sigma$).
L'ensemble de ce qui n'appartient pas � $\Sigma$ est appel� l'univers.

On d�finit la temp�rature comme un param�tre physique qui d�crit le degr� d'agitation thermique. De fa�on g�n�rale, un syst�me thermo�lastique\footnote{Syst�me thermo�lastique : fluide (liquide ou gaz)} d�pend de :
\begin{itemize}
	\item La pression $P$,
	\item La temp�rature $T$,
	\item Le volume $V$.
\end{itemize}

Ce type de syst�me va pouvoir �changer de l'�nergie avec l'ext�rieur. Il existe deux possibilit�s :
\begin{itemize}
	\item Via le travail d'une force : �change d'�nergie "ordonn�e" par l'application de forces ext�rieures sur le syst�me.
	\item Via la chaleur : Il s'agit d'un �change d'�nergie qui s'effectue de fa�on d�sordonn�e au sein du syst�me. Cet �change s'effectuant au sein du syst�me.
\end{itemize}

   
\paragraph{Attention :} Ne pas confondre chaleur (�nergie) et temp�rature (caract�ristique physique).

\section{Caract�ristiques des syst�mes}
\begin{itemize}
	\item Syst�me ouvert : Syst�me pouvant �changer de la mati�re et de l'�nergie avec l'ext�rieur.
	\item Syst�me ferm� : Syst�me pouvant seulement �changer de l'�nergie avec l'ext�rieur.
	\item Syst�me isol� m�caniquement : Syst�me pouvant �changer de l'�nergie uniquement par la chaleur.
	\item Syst�me isol� globalement : Syst�me ne pouvant pas �changer d'�nergie avec l'ext�rieur.
\end{itemize}

\section{�chelles et notion de temp�rature}
\subsection{�chelle Celsius}
C'est une �chelle lin�aire � deux points fixes. Ces points sont :
\begin{itemize}
	\item Temp�rature de fusion de l'eau (0�C)
	\item Temp�rature d'�bullition de l'eau (100�C)
\end{itemize}
Pour une pression de 1 Bar ($\approx10^{5}$Pa).
Dans cette �chelle, on mesurera :
$$T = 100*\frac{x-x_{0}}{x_{100}-x_{0}}$$
Avec $T$ la temp�rature et $x$ la grandeur thermom�trique mesur�e (niveau de mercure par exemple)

\subsection{�chelle thermodynamique}
Si on �tudie la variation de volume $V$ avec la temp�rature Celsius $T$ on trouve $V = V(T)$. Si on extrapole cette relation pour trouver $V=0$, on obtient 
$$T_{abso}=-273,15$$
On d�fini alors la temp�rature thermodynamique par translation : 
$$T_{th} = T_C - T_{abso}$$
$$T_{th} = T_C - 273,15$$

\subsection{Phase d'un syst�me}
Un corp physique peut exister sous trois �tats :
\begin{itemize}
	\item Solide
	\item Liquide
	\item Gazeux
\end{itemize}
\subsection{Point triple d'un corps pur}
�tat dans lequel existent simultan�ment le trois phases du corps donn�
\subsection{Principe 0 de la dynamique}
Lorsque deux syst�mes sont mis en contact thermique, au bout  d'un certain temps ils atteindrons l'�quilibre thermique, caract�rise par une temp�rature finale �gale (et uniforme) pour ces deux corps.
\begin{figure}[h]
	\centering
	\def\svgwidth{10cm}
	\input{fig1_principe0.pdf_tex}
\end{figure}

\section{Capacit� thermique}
\subsection{Cas des solides et des liquides}
On d�finit l'�change thermique �nerg�tique par
$$Q = m.c.\Delta T$$
avec :
\begin{itemize}
	\item[$Q$ :] L'�change thermique �nerg�tique (en Joules)
	\item[$m$ :] La masse du corps
	\item[$\Delta T$ :] $T_{final} - T_{initial}$
	\item[$c$ :] La capacit� thermique massique (en $J.kg^{-1}.K^{-1}$)
\end{itemize}

\paragraph{Attention} Pour utiliser cette formule, on suppose que la capacit� thermique massique ($c$) est constante sur la gamme de temp�rature ($[T_i;T_f]$) �tudi�e.

On d�finit aussi la capacit� calorifique $C$ comme 
$$C=m.c$$
Ce qui donne $Q=C.\Delta T$. On rencontre aussi la notion de capacit� thermique molaire not�e $C_m$. On a alors $Q=n.C_m.\Delta T$ avec $n$ le nombre de moles.

\begin{table}[h]
	\begin{center}
		\begin{tabular}{ c | c}
			Corps Pur & $c$ ($J.kg^{-1}.K^{-1}$) \\
		\hline
			Glace & 2100 \\
			Aluminium & 902 \\
			Diamant & 506 \\
			Argent & 236 \\
		\hline
			Eau & 4185 \\
			Ethanol & 2424\\
			\hline
		\end{tabular}
	\end{center}
	
	\caption{Exemples de valeurs de capacit�s thermiques massiques}
	\label{Exemples de valeurs de capacit�s thermiques massiques}
	
\end{table}

\subsection{�volution en fonction de la temp�rature}
Nous allons illustrer dans le cas o� on utilise la capacit� thermique molaire $C_m$ pour une variation de temp�rature suffisament faible not�e $dT$.

Soit $\delta Q$ le transfert thermique infinit�simal associ� � la variation $dT$.
$$\delta Q=n.C_m(T).dT$$
Pour faire passer le corps �tudi� de la temp�rature $T_i$ � la temp�rature $T_f$, on va utiliser le transfert thermique
\begin{eqnarray*}
Q&=\int^{T_f}_{T_i}\delta Q.dT\\
&=\int^{T_f}_{T_i}n.C_m(T).dT
\end{eqnarray*}

En g�n�ral, $C_m(T)$ est une fonction analytique de la temp�rature.
$$C_m(T) = a_o+\frac{b}{T}+\frac{c}{T^2}$$

\paragraph{Remarque}
On d�finit parfois une capacit� calorifique molaire moyenne par
$$\overline{C_m}=\frac{1}{\Delta T}\int^{T_1}_{T_f}C_m(T).dT$$

\section{Chaleur latente d'un corps pur}
Un changement de phase est un changement de l'�tat physique du corps pur consid�r�. On utilise un diagramme (diagramme de phases) pour repr�senter l'existence de ces �tats physiques. 
\newpage
\begin{figure}[ht]
	\centering
	\def\svgwidth{10cm}
	\input{fig2_diagramme_point_triple.pdf_tex}
	\caption{Exemple de diagramme de phases}
\end{figure}

\paragraph{Cas particulier} Pour l'eau, la courbe de fusion est n�gative (ici en rouge)

Le point triple est le point pour lequel les trois �tats physiques du corps pur �tudi� coexistent.

Les changements de phases se font � pression et temp�rature fixe (pour un corps donn�). Si on consid�re que le corps �tudi� a une masse $m$, le transfert thermique n�cessaire pour le faire passer d'une phase $1$ � une phase $2$ vaut :
$$Q=m.l_{12}$$
O� $l_{12}$ est la chaleur latente massique du changement de phase $1\rightarrow 2$. $l_{12}$ est en $J.K^{-1}$


\chapter{Diffusion thermique}
\section{G�n�ralit�s}
Les transferts thermiques peuvent se r�aliser de trois fa�ons :
\begin{itemize}
	\item Par diffusion
	\item Par convection
	\item Par rayonnement
\end{itemize}
la diffusion est un mode de transfert d'�nergie sans d�placement macroscopique de mati�re. En effet, ce transfert est li� � l'agitation thermique. Ainsi, il s'effectue de proche en proche.

\section{Courant de diffusion thermique}
La diffusion thermique se met en place lorsqu'on rencontre une h�t�rog�n�it� de temp�rature. la diffusion thermique va tendre � uniformiser le champ de temp�rature. On va introduire la notion de courant thermique pour caract�riser ces transferts.

\subsection{Vecteur densit� de courant thermique}
Soit $\overrightarrow{J_Q}$ le vecteur de diffusion thermique tel que son flux � travers une surface $\Sigma$ repr�sente la puissance thermique traversant cette surface.
\newpage
\begin{figure}[h]
	\centering
	\def\svgwidth{5cm}
	\input{fig3_vect_diffusion_th.pdf_tex}
	\caption{Vecteur de diffusion thermique}
\end{figure}

Soit $Q$ la chaleur traversant la surface $\Sigma$.
$$Q=\left[\iint_{\Sigma}\overrightarrow{J_Q}(M).\overrightarrow{dS}\right]dt$$
O� $\overrightarrow{J_Q}(M).\overrightarrow{dS}$ repr�sente le flux �l�mentaire thermique traversant une surface �l�mentaire $d\Sigma$. On le note
$$d\Phi_\Sigma = \overrightarrow{J_Q}(M).\overrightarrow{dS}$$
O� $\overrightarrow{dS}=dS\overrightarrow{n}$\footnote{$\overrightarrow{n}$ est un vecteur unitaire normal (?) � la surface $\Sigma$}. On peut alors d�finir la chaleur �l�mentaire $\delta^2Q$ traversant $\Sigma$.
$$\delta^2Q = \overrightarrow{J_Q}(M).\overrightarrow{dS}.dt$$ 
Pour trouver la chaleur �l�mentaire traversant la surface $\Sigma$ on int�gre la quantit� $\delta^2Q$.

\begin{eqnarray*}
\delta Q&=\int\delta^2Q\\
&=\left[\iint_{\Sigma}\overrightarrow{J_Q}(M).\overrightarrow{dS}\right]dt
\end{eqnarray*}

On d�finit ainsi la puissance thermique � travers $\Sigma$ pendant un temps $dt$.
$$P_{\textsc{th}}=\frac{\delta Q}{dt}=\iint_{\Sigma}\overrightarrow{J_Q}(M).\overrightarrow{dS}$$
avec 
\begin{itemize}
	\item[$P_{\textsc{th}}$] en Watts
	\item[$\overrightarrow{J_Q}$  ] en $W.m^{-2}$
\end{itemize}

\subsection{Loi de Fourrier}
Il s'agit d'une loi ph�nom�nologique obtenue � partir de r�sultats exp�rimentaux.
$$\overrightarrow{J_Q}(M) = -\lambda.\overrightarrow{grad}~T$$
O� $\lambda$ est le coefficient de conductivit� thermique.
\paragraph{Remarque} Le signe $\ominus$ traduit le fait que le transfert thermique s'effectue toujours du point le plus chaud vers le point le plus froid.
Pour un syst�me de coordonn�es cart�siennes, on a :
$$\overrightarrow{grad}~a(x,y,z) = \overrightarrow{e_x}.\frac{\partial a}{\partial x}+\overrightarrow{e_y}.\frac{\partial a}{\partial y}+\overrightarrow{e_z}.\frac{\partial a}{\partial z}$$
\paragraph{Rappel} L'intensit� de $\overrightarrow{J_Q}(M)$ not�e $||\overrightarrow{J_Q}(M)||$ est reli�e � l'amplitude de $||\overrightarrow{grad}~T||$ c'est � dire � l'h�t�rog�n�it� de $T$ dans l'espace 3D
\paragraph{Remarque} On peut observer certaines analogies avec l'�lectrocin�tique. En effet :
\begin{itemize}
	\item On caract�rise le courant des porteurs de charge (les �lectrons) par le vecteur $\overrightarrow{J_C} = \sigma\overrightarrow{grad}~V$ o� $\sigma$ est la r�sistivit�. L'analogie est donc : 
	$$T\longleftrightarrow V$$
	$$\lambda \longleftrightarrow \sigma$$
	\item De m�me, en terme de puissance, on a $P_{\textsc{th}} = \iint \overrightarrow{J_Q}.d\overrightarrow{S}$ � comparer � l'intensit� $I=\iint \overrightarrow{J_C}.d\overrightarrow{S}$
\end{itemize}

\newpage
\section{�quation de diffusion en r�gime stationnaire}
\subsection{D�monstration dans le cas g�n�ral}
On va, par simplification, traiter le cas en une dimension.

\begin{figure}[h]
	\centering
	\def\svgwidth{7.5cm}
	\input{fig4_vect_diffusion_3d.pdf_tex}
	\caption{Vecteur de diffusion thermique en trois dimension}
\end{figure}
On a choisi un parall�l�pip�de parall�le au plan $(Oyx)$ et positionn� sur l'axe $(Ox)$. On obtient : 
$$T(x,y,z,t)\longmapsto T(x,t)$$
On caract�rise le transfert thermique entrant par la face $ABCD$ pendant le temps $dt$ � travers la face $ABCD$.

\begin{eqnarray*}
\delta Q_x &=&- \iint\lambda\left.\frac{\partial T}{\partial x}\right|_x dS_x .\overrightarrow{e_x}.\overrightarrow{e_x}.dt\\
 &=&-\lambda	S \left.\frac{dT(x,t)}{dx}\right|_x dt
\end{eqnarray*}
De m�me, on aura � travers la face de sortie $EFGH$
$$\delta_{x+dx} = \lambda\left.\frac{dT(x,t)}{dx}\right|_{x+dx}S.dt$$
Le gain d'�nergie sera alors donn� par

\begin{eqnarray*}
\delta^2Q &=&\delta Q_x - \delta  Q_{x+dx}\\
&=&-\lambda S\left[ \left. \frac{dT}{dx}\right|_x - \left.\frac{dT}{dx}\right|_{x+dx}\right] dt
\end{eqnarray*}

Or on a $\left.\frac{dT}{dx}\right|_{x+dx} = \left. \frac{dT}{dx}\right|_x + \frac{d}{dx}\left(\frac{dT}{dx}\right)dx$ donc, d'apr�s le d�veloppement de Taylor\footnote{le th�or�me de Taylor appel� aussi la formule de Taylor, montre qu'une fonction plusieurs fois d�rivable au voisinage d'un point peut �tre approxim�e par une fonction polyn�me dont les coefficients d�pendent uniquement des d�riv�es de la fonction en ce point.}\footnote{A VERIFIER, je ne vois pas l'utilit� ni l'application...} on a :
$$\delta^2Q_{gain} = \lambda S\frac{d^2T(x,t)}{d^2x}dx.dt$$

Ce gain d'�nergie calorifique est aussi �gal � la quantit� d�finie � partir de la capacit� calorifique massique : 
$$\delta^2Q = \lambda S\frac{s^2T(x,t)}{d^2x}dx.dt = m.c.dt$$

Or, on remarque que $S.dx$ est le volume du parall�l�pip�de. On pose $\rho$ la masse volumique du solide �tudi�, on a donc $m=\rho.S.dx$. D'o�
\begin{eqnarray*}
	dt.\lambda S\frac{d^2T(x,t)}{d^2x}dx &=&\rho.S.dx.c.dT\\
	\Leftrightarrow dt.\lambda\frac{d^2T(x,t)}{d^2x}&=&\rho.c.dT
\end{eqnarray*}
d'o�
$$	\frac{\partial^2T(x,t)}{\partial^2x}-\frac{c.\rho}{\lambda}.\frac{\partial^2T(x,t)}{\partial^2t} =0$$

On pose $\frac{\lambda}{c\rho}=D_{\textsc{th}}$ o� $D\_{\textsc{th}}$ est le coefficient de diffusion thermique.

\subsection{Cas du r�gime permanent stationnaire}
En r�gime permanent, toutes les variations vis � vis du temps ($t$) seront nulles. D'o�:
$$\frac{\partial T(x,t)}{\partial t}=0$$
Dans ce cas, on cherchera � r�soudre
$$\frac{d^2T(x)}{d^2t}=0$$.
Cette �quation peut aussi s'�crire sous la forme
$$\Delta T(x)=0$$
o� $\Delta$ est un op�rateur nomm� Laplacien\footnote{On le rencontre aussi sous la forme $\nabla^2$}. ici, en coordonn�es cart�siennes, $\Delta = \frac{\partial^2}{\partial^2x}$ pour un probl�me en une dimension. Cependant si on est en trois dimensions, on a :
$$\Delta T(x,y,z) = \frac{\partial^2T(x,y,z)}{\partial^2x} + \frac{\partial^2T(x,y,z)}{\partial^2y}+ \frac{\partial^2T(x,y,z)}{\partial^2z}$$
Si on est dans un autre syst�me de coordonn�es, il faut utiliser l'expression du Laplacien dans ce jeu de coordonn�es.

\section{Solution de l'�quation de la chaleur (r�gime permanent)}
On cherche une fonction de la variable $x$ (dans un probl�me en une dimension ici) qui repr�sente la fonction temp�rature.

la face d'entr�e � $x=0$ est � la temp�rature fixe $T_1$. La face de sortie en $x=L$ est � temp�rature fixe $T_2$. On part de l'�quation $\frac{d^2T(x)}{d^2x}=0$ et on d�termine $T(x)$ par int�grales successives de l'�quation de la chaleur
$$\frac{d^2T(x)}{d^2x}=0\quad\underrightarrow{Integration}\quad\frac{dT(x)}{2x}=A\quad\underrightarrow{Integration}\quad T(x)=Ax+B$$
Or, en $x=0$, $T(0)=T_1$ donc $T(0)=B=T_1$ et en $x=L$, $T-L=T_2$ donc $T(L) = AL+T_1 = T_2$ d'o� $A = \frac{T_2-T_1}{L}$. On a donc
$$T(x)=\frac{T_2-T_1}{2}x+T_1$$

\section{Resistance thermique}
On rappelle que la puissance thermique traversant une surface $\Sigma$ est donn�e par

\begin{eqnarray*}
	P_{\textsc{th}}&=&\iint_\Sigma\overrightarrow{J_Q}.d\overrightarrow{S}\\
	&=&-\lambda\iint_\Sigma\overrightarrow{grad}~T.dS\\
	&=&-\lambda\iint_\Sigma\frac{\partial T}{\partial x}.dS\\
	&=&-\lambda\frac{\partial T}{\partial x}~S
\end{eqnarray*}
Dans l'exemple pr�c�dent, $T(x) = \frac{T_2-T_1}{L}x+T_1$  donc $\frac{\partial T(x)}{\partial x} = \frac{T_2-T_1}{L}$  donc
$$P_{\textsc{th}} = -\lambda\frac{T_2-T_1}{L}~S$$
On pose ici la d�finition de la r�sistance thermique comme �tant 
$$R_{\textsc{th}}= \frac{T_2-T_1}{P_{\textsc{th}}}$$
ou encore $R_{\textsc{th}}= \frac{L}{\lambda S}$

\paragraph{Remarque} Si on associe des conducteurs thermiques en s�rie, la r�sistance thermique totale sera
$$R_{\textsc{th}_{total}}=\sum^{N}_{i=1}R_{\textsc{th}_i}$$
De m�me, si on associe les conducteurs en parall�le, on obtiens
$$\frac{1}{R_{\textsc{th}_{total}}} = \sum^{N}_{i=1}\frac{1}{R_{\textsc{th}_i}}$$

\section{Analogie avec la diffusion particulaire}
Dans la diffusion particulaire, on �tudie la diffusion de particules (atomes, mol�cules) dans l'espace.

On aura un courant de diffusion de particules d�s que la concentration de celles-ci n'est pas homog�ne dans l'espace. On note $\overrightarrow{J_N}$ le vecteur densit� de flux particulaire. On a :
$$dN = \left[\iint\overrightarrow{J_N}.\overrightarrow{dS}\right]dt$$
$dN$ �tant la variation du nombre de particules traversant la section $\Sigma$ �tudi�e pendant le temps $dt$. Le vecteur densit� $\overrightarrow{J_N}$ est donn� par la loi de Fick :
$$\overrightarrow{J_N} = -D\overrightarrow{grad}~n$$
o� $n$ est la densit� de particules (en particules/$m^3$) et $D$ le coefficient de diffusion particulaire (en $m^2.s^{-1}$).

Pour l'�quation de diffusion particulaire, on proc�de comme pour la diffusion thermique, on fait un bilan :
$$D\cdot\Delta n -\frac{\partial n}{\partial t} = 0$$
Donc en r�gime constant, on aura $D.\Delta n=0$


\chapter{Transformation thermodynamique et gaz parfaits}
\section{�quation d'�tat d'un syst�me thermodynamique}
\subsection{Variable d'�tat}
Il s'agit d'une variable macroscopique mesurable et d�crivant le syst�me dans son ensemble. Par exemple $P$,$V$ et $T$
\subsection{Fonction d'�tat}
C'est une fonction qui relie entre elles les diff�rentes variables d'�tat. Elle �tablit $f(P,V,T)=0$

\section{�quilibre et transformation d'un syst�me}
On se basera sur un syst�me globalement isol� et stationnaire pour lequel $\frac{\partial X}{\partial t}=0$. Ce syst�me est donc � l'�quilibre lorsque ses variables sont aussi � l'�quilibre thermodynamique. C'est-�-dire lorsqu'on a
$$\frac{df(P,V,T)}{dt}=0$$
\subsection{Transformations thermodynamiques}
Il s'agit d'�volution possible qui fait passer l'�tat thermodynamique d'un syst�me d'un �tat initial stable � un autre �tat, final, lui aussi stable.
\subsubsection{Transformation quasi-statique}
Transformation pour laquelle les �tats interm�diaires sont des �tats d'�quilibre infiniment voisins.
Ces transformations se r�alisent tr�s lentement.
\subsubsection{Transformations r�versibles et irr�versibles}
Les transformations r�versibles sont des transformations pour lesquelles on peut revenir � l'�tat initial � partir de l'�tat final en effectuant une transformation "inverse"

Une transformation irr�versible est donc une transformation qui ne permet pas se retour en arri�re.

\paragraph{Remarque} Dans la nature, toute transformation est irr�versible, notamment � cause des frottements.
\paragraph{Remarque} Toute transformation r�versible est quasi-statique mais l'inverse est faux.

\subsubsection{Transformation isotherme}
La temp�rature du syst�me est constante et est celle du milieu ext�rieur.
$$T(t)=T_{ext}=cste$$
\subsubsection{Transformation monotherme}
La temp�rature finale du syst�me est la m�me que la temp�rature initiale.
$$T_f=T_i$$
\subsubsection{Transformation isobare/monobare}
De m�me que les transformations isotherme et monotherme mais en fixant la pression au lieu de la temp�rature.
\subsubsection{Transformation cyclique}
Transformation revenant � son �tat initial au bout d'un certain temps.
\subsubsection{Transformation adiabatique}
Transformation qui s'effectue sans transfert thermique avec l'ext�rieur.
$$\delta Q = 0$$
\subsection{Thermostat}
Il s'agit d'une source thermique id�ale. Un syst�me pour lequel on peut consid�rer sa temp�rature comme une constante.
$$T=T_{init}=cste$$
En pratique, on consid�re qu'un thermostat est un corps caract�ris� par une capacit� thermique quasi-infinie.
\subsection{Variables extensives et intensives}
Une variable extensive est une variable additive sous l'association de plusieurs sous-syst�mes.
Une variable intensive est une variable locale dans le sous-syst�me.
\section{Mod�le du gaz parfait}
Un gaz parfait est un gaz suffisamment dilu� pour pouvoir n�gliger les interactions entre les particules qui le composent � l'exeption de l'instant o� elles entres en collision\footnote{Th�orie cin�tique des gaz}
\subsection{�quation d'�tat du gaz parfait}.
$$PV=nRT$$
Avec : 
\begin{itemize}
	\item [P :] la pression
	\item [V :] le volume
	\item [n :] le nombre de moles de gaz
	\item [R :] la constante des gaz parfaits
	\item [T :] la temp�rature
\end{itemize}

\section{Travail des forces de pression}
\subsection{Travail infinit�simal}
On consid�re un gaz sous une pression $P$ contenu dans un r�cipient cylindrique de section $S$.
\begin{figure}[h]
	\centering
	\def\svgwidth{10cm}
	\input{fig8_travail_infinitesimal.pdf_tex}
	\caption{Travail infinit�simal}
\end{figure}
On a ici :
$$\overrightarrow{F_{ext}} = -P_{ext}.S.\overrightarrow{i}$$
On d�finit le travail �l�mentaire par
\begin{eqnarray*}
	\delta W &=& \overrightarrow{F_{ext}} . d\overrightarrow{l}\\
	&=&-P_{ext} . S.dl . \overrightarrow{i} . \overrightarrow{i}\\
	&=&-P_{ext} . dV
\end{eqnarray*}
\paragraph{Remarque}
\begin{itemize}
	\item Si $dV < 0$ alors on r�alise une compression  du gaz et $\delta w > 0$. Le gaz re�oit de l'�nergie.
	\item Si $dV > 0$ alors on r�alise une d�tente du gaz et $\delta w < 0$. Le gaz perd de l'�nergie.
\end{itemize}
\subsection{Cas particulier des transformations quasi-statiques}
Pour une transformation quasi-statique, chaque �tat interm�diaire de la transformation est un �tat d'�quilibre thermodynamique donc la pression ext�rieure $P_{\textsc{ext}}$ est �gale � la pression du gaz. On a donc :
$$P = P_{ext}$$
Dans ce cas, le travail �l�mentaire des forces de pression est
$$\delta W = -PdV$$
\subsection{Travail global}
On a, d'une mani�re g�n�rale :
$$W_{A\rightarrow B} = \int_A^B\delta W$$
Soit, dans le cas d'une r�action quasi-statique : 
$$W_{A\rightarrow B} = -\int_A^BPdV$$
Ou, dans le cas d'une transformation non quasi-statique :
\begin{eqnarray*}
	W_{A\rightarrow B} &=& -\int_A^BP_{ext}.dV\\
	&=&-P_{ext} \int_A^B. dV
\end{eqnarray*}
Dans le cas o� $P_{ext}$ est constante.
\chapter{Premier principe de la thermodynamique}
Un syst�me est caract�ris� par ses variables d'�tat. Ce syst�me peut �voluer entre deux �tats d'�quilibres (par exemple $A$ et $B$). Le premier principe de la thermodynamique a pour but de faire le bilan �nerg�tique d'un syst�me.
\section{�nergie totale et �nergie interne}
On rappelle qu'un syst�me est caract�ris� par son �nergie microscopique.
$$E_m = E_c + E_p$$
On rappelle que les forces conservatives d�rivent d'une �nergie potentielle
$$\overrightarrow{F_c} = -\overrightarrow{grad}~E_p$$
On va g�n�raliser le th�or�me de l'�nergie microscopique en introduisant l'�nergie interne. Pour tout syst�me form�, on d�finit l'�nergie totale $E_{tot}$ par $E_{tot} = E_c + E_p + U$ o� $U$ d�signe l'�nergie interne du syst�me\footnote{Cours peut-�tre incomplet, A VERIFIER}. De cette fa�on, en l'absence d'�change d'�nergie avec l'ext�rieur, on aura $E_{tot}$ constante.
\section{Bilan d'�nergie et $1^{er}$ principe}
Pour un syst�me pouvant �changer de l'�nergie avec l'ext�rieur, il peut le faire de deux fa�ons :
\begin{itemize}
	\item Par le travail des forces m�caniques ($W$)
	\item Par le transfert thermique ($Q$)
\end{itemize}
Le bilan �nerg�tique d'un syst�me sera :
\begin{eqnarray*}
	\Delta E_{tot} &=& E_{tot}(final) - E_{tot}(init)\\
	&=&-\Delta E_m + \Delta Q\\
	&=& W + Q
\end{eqnarray*}
\paragraph{Remarque} $W$ et $Q$ sont alg�briques, si $W$ ou $Q$ est n�gatif, il y a une perte d'�nergie. Sinon, il y a un gain.
Dans beaucoup de cas, on a un syst�me pour lequel on peut consid�rer que $E_c + E_p = 0$ (ou au moins $\Delta E_m < 0$). Dans ce cas, on a
$$\Delta U = W + Q$$
Si on effectue une transformation infinit�simale, on aura :
$$dU = \delta W + \delta Q$$
$U$ est une fonction d'�tat dont la variation ne d�pend que des �tats initial et final et non du chemin suivi pour y arriver. Par cons�quent, $dU$ est une diff�rentielle totale exacte. On peut la noter :
$$\sum_{i=1}^n\left(\frac{\partial U}{\partial X_i}\right)dx$$
Par exemple, $dU = \left(\frac{\partial U}{\partial V}\right)dV + \left(\frac{\partial U}{\partial T}\right)dT$ pour le couple de variables $(V;T)$. Par contre, $\delta W$ et $\delta Q$ ne sont pas des diff�rentielles totales exactes. Leurs variations �l�mentaires d�pendent du chemin suivi.
\section{Application du $1^{er}$ principe � la d�tente de Joule}
On consid�re :
\begin{figure}[h]
	\centering
	\def\svgwidth{\columnwidth}
	\input{fig9_detente_joule.pdf_tex}
	\caption{D�tente de Joule}
\end{figure}

On supposera les parois adiabatiques ($\delta Q=0$) et ind�formables ($\delta W = 0$). Pour une telle d�tente\footnote{Appel�e "d�tente de Joule-Gay-Lussac"} on aura :
$$dU = \Delta U = 0$$
On en d�duit que $U$ reste constante.

\section{$1^{�re}$ loi de Joule}
Pour un gaz parfait, l'�nergie interne $U$ n'est fonction que de la temp�rature.
$$U = U(T)$$
Donc, pour un gaz parfait, on a $\frac{\partial U}{\partial V} = 0$

\section{Capacit� calorifique des gaz � volume constant}
Pour un gaz parfait, on a :
$$C_v = \frac{\partial U}{\partial T}$$
Or, d'apr�s la loi de Joule,
$$dU = \frac{\partial U}{\partial T}$$
donc $dU = C_.dT$ et $\Delta U = C_v\Delta T$ pour $C_v$ ind�pendant de la temp�rature.

\paragraph{Remarque} Si $C_v$ d�pend de la temp�rature, alors
$$\Delta U = \int C_v(T).dT$$

Pour un gaz parfait monoatomique, on peut montrer que $C_v = \frac{3}{2} nR$ et $U(T) = \frac{3}{2}nRT$. De m�me, pour un gaz diatomique, $C_v = \frac{5}{2}nR$ et $U(T) = \frac{5}{2}nRT$

\section{Application du premier principe aux transformations thermodynamiques}
On va illustrer  le cas d'une transformation isotherme d'un gaz parfait d'un �tat $A$ � un �tat $B$.

Le gaz parfait est d�fini par son �quation d'�tat $PV = nRT$. Puisque la r�action est isotherme, $T_A = T_B$. On en d�duit donc que :
$$P_AV_A = P_BV_B$$
Si la transformation est quasi statique, le travail des forces de pression est :
\begin{eqnarray*}
	\delta W &=& -PdV \\
	&=& -\frac{nRT}{V}dV\\
	\Rightarrow W &=& -nRT\int_A^B\frac{dV}{V}\\
	&=& -nRT. ln(\frac{V_A}{V_B})
\end{eqnarray*}

\paragraph{Remarque} On peut aussi exprimer le travail en fonction de la pression $P$. On note que $P.V$ reste constant au cours de la transformation.

\section{Enthalpie et capacit� calorifique des gaz � pression constante}
Une transformation isobare s'effectue � pression constante. On cherche � �valuer le transfert thermique $\delta Q_p$ lors d'une telle transformation.

On peut poser $dP = 0$ d'o�\footnote{A VERIFIER}:
\begin{eqnarray*}
	PdV &=& P.dV + \overbrace{V.dP}^{=0}\\
	&=& d(PV)
\end{eqnarray*}
Par ailleur, on a :
\begin{eqnarray*}
	dU &=& \delta W + \delta Q_p \\
	\Leftrightarrow dU &=& -P.dV + \delta Q_p \\
	\Leftrightarrow dU + d(PV)&=&  \delta Q_p \\
	\Leftrightarrow d(U + P.V) &=& \delta Q_p \\
	&=&dH\\
\end{eqnarray*}
O� $H = U+PV$ repr�sente l'enthalpie qui est aussi une fonction d'�tat (tout comme $U$). On d�duit alors que $\Delta H=Q_p$. De plus, on peut exprimer la diff�rentielle de $H$ en variables $T$ et $P$.
$$dH = \frac{\partial H}{\partial T} dT + \frac{\partial H}{\partial P} dP$$
On pose
$$\frac{\partial H}{\partial T} = C_p$$
\section{$2^{�me}$ loi de Joule}
Sachant qu'un gaz parfait ob�it � la premi�re loi de Joule ($U = U(T)$) et sachant qu'on a :
\begin{eqnarray*}
	H = U+P.V &=&U+nRT\\
	&=& U(T) + nRT
\end{eqnarray*}
On constate qu'un gaz parfait ob�it � la deuxi�me loi de Joule. Son enthalpie est fonction de la seule variable temp�rature.
$$H = H(T) $$
Donc, pour un gaz parfait, on aura $\frac{\partial H}{\partial P} = 0$

\section{Relation de Mayer}
\begin{eqnarray*}
	H &=& u+PV\\
	&=& U + nRT
\end{eqnarray*}
donc
\begin{eqnarray*}
	\frac{dH}{dT} & = & \frac{dU}{dT} + nR\\
	\frac{\partial H}{\partial T} & = & \frac{\partial U}{\partial T} + nR\\
	C_p &=& C_v + nR\\
	C_p - C_v &=& nR
\end{eqnarray*}

\paragraph{Remarques} Soit $C_{p;m}$ la capacit� calorifique molaire � pression constante. On a $C_{p;m} = \frac{C_p}{n}$ d'o� $C_{p;m} - C_{v;m} = R$

On d�finit aussi le rapport
$$\gamma = \frac{C_p}{C_v}$$

a partir de la relation de Mayer, on d�termine $C_p$
\begin{itemize}
	\item Gaz monoatomique
			\begin{eqnarray*}
					C_p &=& C_v + nR\\
					&=& \frac{3}{2}nR + nR\\
					 &=& \frac{5}{2}nR
			\end{eqnarray*}
	\item Gaz diatomique
	\begin{eqnarray*}
					C_p &=& C_v + nR\\
					&=& \frac{5}{2}nR + nR\\
					&=& \frac{7}{2}nR
			\end{eqnarray*}
\end{itemize}

\section{Repr�sentation graphique des transformations}
Pour repr�senter des transformations thermodynamiques, on utilise le diagramme de Clapeyron.
\begin{figure}[h]
	\centering
	\def\svgwidth{7cm}
	\input{fig10_clapeyron_simple.pdf_tex}
	\caption{Exemple de diagramme de Clapeyron}
\end{figure}
L'int�grale $\int_{V_A}^{V_B} P.dV$ repr�sente l'aire sous la courbe $P=f(V)$. Cette int�grale est une grandeur alg�brique.
\begin{itemize}
	\item Si $ V_A > V_B$ alors on a 
	$$\int_{V_A}^{V_B}P.dV < 0$$
	Dans ce cas, 
	$$W_{A\rightarrow B} = -\int_{V_A}^{V_B} P.dV > 0$$
	\item Si $ V_A < V_B$ alors on a 
	$$\int_{V_A}^{V_B}P.dV > 0$$
	Dans ce cas, 
	$$W_{A\rightarrow B} = -\int_{V_A}^{V_B} P.dV < 0$$
\end{itemize}

\subsubsection{Cas particulier des transformations cycliques}
\begin{figure}[h]
	\centering
	\def\svgwidth{7cm}
	\input{fig11_clapeyron_reversible.pdf_tex}
	\caption{Diagramme de Clapeyron pour une transformation r�versible}
\end{figure}
Sur le cycle, on a $W_{cycle} = W_{A\rightarrow B} + W_{B\rightarrow A}$. De plus, on observe que
$$|W_{A\rightarrow B}| > |W_{B\rightarrow A}|$$
Or, on a $W_{A\rightarrow B} < 0 $ donc on aura $W_{cycle} < 0$
\section{Transformation adiabatique r�versible et loi de Laplace}
On exprime le premier principe de la thermodynamique :
\begin{eqnarray*}
					dU &=&  \delta Q + \delta W\\
					\Leftrightarrow \delta Q &=& dU - \delta W \\
					&=& 0
\end{eqnarray*}
Car la transformation est adiabatique. Pour un gaz parfait, on a
\begin{eqnarray*}
					dU &=&  C_V.dT\\
					0 &=& C_V.dT - \delta W \\
					&=& nC_{V,m}dT+P.dV
\end{eqnarray*}
Car $C_V =\frac{nR}{\gamma -1}$. Or $PV = nRT$ donc on a :
$$0 = \frac{nRdT}{\gamma -1}+nRT\frac{dV}{V}$$
Et, si on divise par $T$,
\begin{eqnarray*}
				0 &=& \frac{nR}{\gamma -1} \cdot \frac{dT}{T} + nR\cdot\frac{dV}{V}\\
				 &=& \frac{dT}{T} + (\gamma -1) \frac{dV}{V}
\end{eqnarray*}
On int�gre pour obtenir 
\begin{eqnarray*}
				K &=& ln(T) + ln(V^{\gamma -1})\\
				&=& ln(T.V^{\gamma -1})
\end{eqnarray*}
O� $K$ est une constante. Donc, $TV^\gamma$ est une constante. En appliquant la loi de Laplace, en coordonn�es $(T,V)$ on obtient $PV^\gamma$ est une constante.
\paragraph{Remarque} Il est possible de lier $TV^\gamma$ et $PV^\gamma$ par :
$$TV^{\gamma-1} = \frac{PV}{nR}V^{\gamma-1}$$
Car on sait que $TV^{\gamma-1}$ est constant donc $\frac{PV}{nR}V^{\gamma-1}$ est constant aussi, donc $\frac{P}{nR}V^{\gamma}$ est constant et donc $PV^{\gamma}$ est constant.

\subsection{Travail re�u par un gaz parfait}
Comme ce type\footnote{Lequel ?} de transformation est adiabatique, on a $\Sigma Q = 0$ et donc $\delta Q = 0$. On peut donc appliquer la premi�re loi de Joule
$$dU = C_VdT = nC_{V,m}dT$$
Et, en appliquant le premier principe de la thermodynamique, on obtient
\begin{eqnarray*}
				dU &=& \delta W + \delta Q\\
				&=& \delta W + 0
\end{eqnarray*}
Or on a $W = \int{\delta W} = \int_A^B{nC_{V,m}dT}$ donc on a
\begin{eqnarray*}
				W &=&nC_{V,m}(T_B-T_A)\\
				&=& C_V(T_B-T_A)
\end{eqnarray*}
Or $C_V=\frac{nR}{\gamma -1}$ donc on a
$$W =\frac{nR}{\gamma -1}(T_B-T_A)$$
Ou encore
$$W =\frac{nR}{\gamma -1}(P_BV_B-P_AV_A)$$

\chapter{Introduction aux gaz r�els}
\section{D�finition}
Pour un gaz parfait, on avait n�glig� les interaction entre les mol�cules ou atomes constituants le gaz.

Pour un gaz r�el, cette hypoth�se n'est plus valide. On va consid�rer l'int�raction mol�culaire/atomique.

Dans un gaz parfait, pour mod�liser les interactions mol�culaires, on utilise "Le mod�le des sph�res dures"
\begin{figure}[h]
	\centering
	\def\svgwidth{10cm}
	\input{fig12_repulsion_parfait.pdf_tex}
	\caption{Mod�le des sph�res dures}
\end{figure}
Ici, $E_p(r)$ repr�sente la r�pulsion entre deux mol�cules en fonction de la distance qui les s�pare.

Avec les gaz r�els, on utilisera plut�t des mod�les comme celui-ci, plus r�aliste.
\begin{figure}[h]
	\centering
	\def\svgwidth{10cm}
	\input{fig12_repulsion_reel.pdf_tex}
	\caption{Mod�le des gaz r�els}
\end{figure}

\section{�quation d'�tat d'un gaz de Van der Waals}
$$P =\frac{nRT}{V-nb} = \frac{an^2}{V^2}$$
ou encore
$$\left(P+\frac{an^2}{v^2}\right)\left(V-nb\right) = nRT$$
Avec $\frac{an^2}{v^2}$ le terme correctif de pression interne et $nb$ le terme correctif de volume\footnote{Ce terme est du � la pr�sence des mol�cules}.
\section{Coefficients thermo�lastiques}
\subsection{Coefficient de dilatation isobare}
Le coefficient de dilatation isobare $\alpha$ vaut :
$$\alpha = \frac{1}{v}\cdot\frac{\partial V}{\partial T}$$
\subsection{Coefficient de compressibilit� isotherme}
Le coefficient de dilatation isobare $\chi_T$ vaut :
$$\chi_T = -\frac{1}{v}\cdot\frac{\partial V}{\partial P}$$
%\part{GLPH202}

\chapter{Le second principe de la thermodynamique}

\section{D�finitions}

	\paragraph{Source de chaleur}Une source de chaleur est un syst�me ferm� n'�changeant de l'�nergie que par transfert thermique.
	\paragraph{Thermostat}Un thermostat est une source de chaleur dont la temp�rature est constante ($Q\longrightarrow+\infty$)

\section{D�finition de l'entropie}

	On rappelle que le premier principe de la thermodynamique donne
	$$dU = \delta Q+\delta W$$
	Dans le cas d'une transformation r�versible (Avec $-PdV=\delta W$). Si on consid�re une transformation r�versible on peut d�finir le transfert thermique par :
	$$\delta Q = TdS$$
	O� $S$ est une fonction d'�tat, extensive et non additive, nomm�e entropie. L'entropie est donn�e en $J.K^{-1}$.

\section{�nonc� du second principe de la thermodynamique}

Pour tout syst�me ferm� en contact avec une ou plusieurs sources de chaleur, on peut d�finir une fonction d'�tat extensive et non additive $S$ telle que
$$dS = \delta S_e + \delta S_c$$
o� $\delta S_e$ est l'entropie �chang�e avec la ou les sources de chaleur avec lesquelles le syst�me est en contact. On pourra la d�finir par 
$$\delta S_e = \frac{\delta Q_e}{T_{e}}$$
O�
\begin{itemize}
	\item $\delta Q_e$ est le transfert thermique re�u par le syst�me.
	\item $T_{e}$ est la temp�rature de la source de chaleur.
\end{itemize}
Et o� $\delta S_c$ est l'entropie cr��e avec
$$\delta S_c \geq 0$$
telle que $\delta S_c = 0$ pour une transformation r�versible et $\delta S_c > 0$ si la transformation est irr�versible.

On peut donner une formule int�gr�e du second principe :
$$\Delta S = S_e - S_c = S_f-S_i$$
Avec
$$S_e = \int_{A}^{B}{\delta S_e} = \int_{A}^{B}{\frac{\delta Q}{T_e}}$$

\section{Causes possibles de l'irr�versibilit�}
\subsection{Forces de frottement}loi de fourrier
Lors d'un frottement, l'�nergie m�canique du syst�me n'est pas conserv�e. Il y a donc irr�versibilit�.

\subsection{�changes thermiques}
Lors du processus de diffusion thermique, le courant de diffusion thermique est irr�versible (loi de Fourrier)

\subsection{M�lange de deux gaz}
Lors du m�lange de deux gaz, il y a diffusion selon la loi de Fick. Ce qui induit, de la m�me fa�on que pour la loi de Fourrier, une irr�versibilit�.

\section{Exemple : d�tente de Joule-Gay-Lussac}

On travaille avec des parois adiabatiques et rigides. On a donc $W=0$ et $Q=0$ donc $\Delta U=0$.

Pour ce qui est de l'entropie, on a $\Delta S =S_e + S_c$ or, les parois sont adiabatiques\footnote{Les parois ou le syst�me ?} donc $S_e=0$. Par ailleurs, $S_c > 0$ car on travaille sur une r�action irr�versible.

On va pouvoir calculer $\Delta S$ ind�pendamment en consid�rant que c'est une fonction d'�tat, donc en consid�rant qu'il existe un chemin r�versible amenant au m�me �tat final.

Dans ce cas, on a $dS = \frac{\delta Q_{reversible}}{T}$ or, on a aussi $\delta Q_{rev} = -\delta W = PdV$ car $\Delta U = 0$. Or on travaille sur un gaz parfait, donc $P=\frac{nRT}{V}$ et $PdV=nrt\frac{dV}{V}$ donc $dS=nR\frac{dV}{V}$ et on a :
$$\Delta S = \int{dS} = nR\int_{V_i}^{V_f}{\frac{dV}{V}} = nR\ln{\frac{V_f}{V_i}}$$

\paragraph{Remarque :}Dans le cas d'une d�tente, on a toujours $V_f > V_i$ donc $\ln{\frac{V_f}{V_i}} > 0$ et donc $\Delta S >0$




%\appendix
%Chapitres annexes
%\bibliographystyle{} % Le style est mis entre crochets.
%\bibliography{bibli} % Mon fichier de base de donn�es s'appelle bibli.bib.

%\backmatter

%Epilogue

\listoffigures
%\listoftables
%\printindex

\end{document}