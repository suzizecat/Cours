\chapter{Principaux théorèmes}
\section{Théorème de Stokes}
\subsection{Énoncé}
Soit $\vec{U}$ un champ vectoriel et $\CCr$ une courbe \emph{fermée et orientée}. Soit $\SS$ une surface qui s'appuie sur $\CCr$ et qui est orientée par $\CCr$ dans le sens direct. On a alors :
\[
\oint_\CCr\vec{U}\circ\der\vec{l} = \iint_\SS(\Rot\vec{U})\circ\der\vec{S}
\]

\subsection{Applications}
Appliqué au champ $\vec{E}$, le théorème de Stokes nous donne
\[\oint_\CCr\vec{E}\circ\der\vec{l}\]

\section{Théorème d'Ostrogradsky}
\subsection{Énoncé}
Soit $\vec{U}(x,y,z)$ un champs de vecteurs et $\SS$ une surface fermée dirigée vers l’extérieur et entourant un volume $\VV$. On a alors :
\[
\oiint_\SS\vec{U}\circ\der\vec{S} = \iiint_\VV\Div(\vec{U})\der\tau
\]